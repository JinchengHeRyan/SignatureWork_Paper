% Options for packages loaded elsewhere
\PassOptionsToPackage{unicode}{hyperref}
\PassOptionsToPackage{hyphens}{url}
%
\documentclass[
]{article}
\usepackage{amsmath,amssymb}
\usepackage{lmodern}
\usepackage{iftex}
\ifPDFTeX
\usepackage[T1]{fontenc}
\usepackage[utf8]{inputenc}
\usepackage{textcomp} % provide euro and other symbols
\else % if luatex or xetex
\usepackage{unicode-math}
\defaultfontfeatures{Scale=MatchLowercase}
\defaultfontfeatures[\rmfamily]{Ligatures=TeX,Scale=1}
\fi
% Use upquote if available, for straight quotes in verbatim environments
\IfFileExists{upquote.sty}{\usepackage{upquote}}{}
\IfFileExists{microtype.sty}{% use microtype if available
    \usepackage[]{microtype}
    \UseMicrotypeSet[protrusion]{basicmath} % disable protrusion for tt fonts
}{}
\makeatletter
\@ifundefined{KOMAClassName}{% if non-KOMA class
    \IfFileExists{parskip.sty}{%
        \usepackage{parskip}
    }{% else
        \setlength{\parindent}{0pt}
        \setlength{\parskip}{6pt plus 2pt minus 1pt}}
}{% if KOMA class
    \KOMAoptions{parskip=half}}
\makeatother
\usepackage{xcolor}
\IfFileExists{xurl.sty}{\usepackage{xurl}}{} % add URL line breaks if available
\IfFileExists{bookmark.sty}{\usepackage{bookmark}}{\usepackage{hyperref}}
\hypersetup{
    hidelinks,
    pdfcreator={LaTeX via pandoc}}
\urlstyle{same} % disable monospaced font for URLs
\usepackage{longtable,booktabs,array}
\usepackage{calc} % for calculating minipage widths
% Correct order of tables after \paragraph or \subparagraph
\usepackage{etoolbox}
\makeatletter
\patchcmd\longtable{\par}{\if@noskipsec\mbox{}\fi\par}{}{}
\makeatother
% Allow footnotes in longtable head/foot
\IfFileExists{footnotehyper.sty}{\usepackage{footnotehyper}}{\usepackage{footnote}}
\makesavenoteenv{longtable}
\usepackage{graphicx}
\makeatletter
\def\maxwidth{\ifdim\Gin@nat@width>\linewidth\linewidth\else\Gin@nat@width\fi}
\def\maxheight{\ifdim\Gin@nat@height>\textheight\textheight\else\Gin@nat@height\fi}
\makeatother
% Scale images if necessary, so that they will not overflow the page
% margins by default, and it is still possible to overwrite the defaults
% using explicit options in \includegraphics[width, height, ...]{}
\setkeys{Gin}{width=\maxwidth,height=\maxheight,keepaspectratio}
% Set default figure placement to htbp
\makeatletter
\def\fps@figure{htbp}
\makeatother
\setlength{\emergencystretch}{3em} % prevent overfull lines
\providecommand{\tightlist}{%
    \setlength{\itemsep}{0pt}\setlength{\parskip}{0pt}}
\setcounter{secnumdepth}{-\maxdimen} % remove section numbering
\ifLuaTeX
\usepackage{selnolig}  % disable illegal ligatures
\fi

\author{}
\date{}

\begin{document}

    \textbf{Duke Kunshan University}

    \textbf{Division of Natural and Applied Sciences}

    \textbf{SW theses - Template}

    Instructions

    \begin{itemize}
        \item
        Black text -- \emph{Do not delete}. Everything in black stays in the
        document.
        \item
        Red text -- \emph{Model text}. Replace with your own text, then change
        the ink color to black.
        \item
        Blue text -- \emph{Delete}. The blue text is instructional, providing
        general guidance for what goes into a particular section.
    \end{itemize}

    \textbf{Overview}

    The Title Page, Table of Contents, and Abstract, are the first three
    pages of the document. The Table of Contents as (iiI), and the Abstract
    as (ii). (The title page does not include a page number). You should
    include a single-page Acknowledgements/Dedication and/or Appendix/ces as
    options. The acknowledgements, or dedication, page is should be the
    second page, numbered (iii), with the Table of Contents becoming (iv).
    The appendix/ces or supplemental content pages should follow the
    source/works cited. The appendix or supplemental content pages should
    continue with the Arabic numbering. All page numbers must be bottom
    centered. See next section for an example outline of the document.

    Title page

    \begin{itemize}
        \item
        Title written in ALL CAPS
        \item
        The title is centered at the top of the page
        \item
        The title cannot exceed three lines
        \item
        The word `by' is on its own line
        \item
        Your name should be capitalized in its regular way
        \item
        The phrases `Signature Work submitted for' are to remain as is
        \item
        Enter the date of submission with normal capitalization as Month, Day,
        Year
    \end{itemize}

    Table of Contents

    \begin{itemize}
        \item
        The table of contents should be left justified, each chapter should be
        noted, including the page number.
    \end{itemize}

    \textbf{REMEMBER TO DELETE THIS PAGE BEFORE SUBMISSION}

    THESIS TITLE THAT EXTENDS OVER ONE LINE GOES IN INVERTED PYRAMID FORM

    by

    First Name\_Last Name

    Signature Work Product, in partial fulfilment of the Duke Kunshan
    University Undergraduate Degree Program

    \emph{\{Enter date of submission with normal capitalization as Month,
        Day, Year\}}

    Signature Work Program

    Duke Kunshan University

    APPROVALS

    \_\_\_\_\_\_\_\_\_\_\_\_\_\_\_\_\_\_\_\_\_\_\_\_\_\_\_\_\_\_\_\_\_\_\_\_\_\_\_\_\_\_\_\_\_\_\_\_\_\_\_\_\_\_\_\_\_\_\_

    \emph{Mentor: First Name, Last Name and Division (without abbreviations)
        using normal capitalization}

    \_\_\_\_\_\_\_\_\_\_\_\_\_\_\_\_\_\_\_\_\_\_\_\_\_\_\_\_\_\_\_\_\_\_\_\_\_\_\_\_\_\_\_\_\_\_\_\_

    Marcia B. France, Dean of Undergraduate Studies

    \protect\hypertarget{_Toc93021168}{}{}\textbf{ABSTRACT} \emph{(in
    English)}

    \emph{150 -- 200 words}. \emph{An abstract is a brief statement of the
    problem or the purpose of the research. It should indicate the
    theoretical work or experimental plan used, summarize principal findings
    of the research, and point out major conclusions. Appropriate safety
    information should be included when applicable. This should be the
    section you write last to be sure that it accurately reflects the
    content of the document.}

    \protect\hypertarget{_Toc93021169}{}{}\textbf{ABSTRACT} \emph{(in
    Chinese)}

    150 - 200
    字。摘要是对问题或研究目的的简要说明。说明所使用的理论工作或实验计划,总结研究的主要发现,并指出主要结论。适用时应包括适当的安全信息。这应该是您最后编写的部分,以确保它准确反映文档的内容。

    150 - 200 Zì. Zhāiyào shi duì wèntí huò yánjiū mùdì de jiǎnyào shuōmíng.
    Shuōmíng suǒ shǐyòng de lǐlùn gōngzuò huò shíyàn jìhuà, zǒngjié yánjiū
    de zhǔyào fāxiàn, bìng zhǐchū zhǔyào jiélùn. Shìyòng shí yīng bāokuò
    shìdàng de ānquán xìnxī. Zhè yīnggāi shì nín zuìhòu biānxiě de bùfèn, yǐ
    quèbǎo tā zhǔnquè fǎnyìng wéndàng de nèiróng.

    \hypertarget{acknowledgements}{%


        \section{ACKNOWLEDGEMENTS}\label{acknowledgements}}

    \emph{Individuals and organizations who helped with the research project
    and provided financing are thanked in a paragraph of the thesis. Do not
    include individual titles in the acknowledgments. However, it is
    appropriate to state grant numbers and sponsors. Examples would like
    SELF, SRS, SW Grants, etc.}

    \hypertarget{table-of-content}{%


        \section{TABLE OF CONTENT}\label{table-of-content}}

    \hypertarget{section}{%


        \section{}\label{section}}

    \protect\hyperlink{_Toc93021168}{\textbf{ABSTRACT 3}}

    \protect\hyperlink{_Toc93021169}{\textbf{ABSTRACT 3}}

    \protect\hyperlink{acknowledgements}{\textbf{ACKNOWLEDGEMENTS 4}}

    \protect\hyperlink{table-of-content}{\textbf{TABLE OF CONTENT 5}}

    \protect\hyperlink{list-of-tables}{\textbf{LIST OF TABLES 6}}

    \protect\hyperlink{table-of-figures}{\textbf{TABLE OF FIGURES 7}}

    \protect\hyperlink{section-1}{\textbf{INTRODUCTION 8}}

    \protect\hyperlink{material-and-methods}{\textbf{MATERIAL AND METHODS
    9}}

    \protect\hyperlink{results}{\textbf{RESULTS 10}}

    \protect\hyperlink{discussion}{\textbf{DISCUSSION 11}}

    \protect\hyperlink{conclusions}{\textbf{CONCLUSIONS 12}}

    \protect\hyperlink{references}{\textbf{REFERENCES 13}}

    \protect\hyperlink{appendices}{\textbf{APPENDICES 14}}

    \protect\hyperlink{_Toc93021181}{\textbf{APPENDIX A: 15}}

    \hypertarget{list-of-tables}{%


        \section{LIST OF TABLES}\label{list-of-tables}}

    \protect\hyperlink{_Toc93020061}{Table 1 Parameters for the optimization
    of the principal component analysis for olive oil adulteration 10}

    \emph{First add captions to your Tables. Right-click on the text above
    and select Update Field to update this list. Word then searches the
    document for your captions and automatically adds a list of tables,
        sorted by page number. The captions must be formatted as in the DNAS SW
        Style Guide}.

    \hypertarget{table-of-figures}{%


        \section{TABLE OF FIGURES}\label{table-of-figures}}

    \protect\hyperlink{_Toc93020444}{Figure 1 The notorious BTC (Brandon the
    Cat) 10}

    \emph{First add captions to your figures. Right-click on the text above
    and select Update Field to update this list. Word then searches the
    document for your captions and automatically adds a list of figures,
        sorted by page number. The captions must be formatted as in the DNAS SW
        Style Guide}.

    \hypertarget{section-1}{%


        \section{}\label{section-1}}

    \hypertarget{introduction}{%


        \section{INTRODUCTION}\label{introduction}}

    \emph{This section includes a clear statement of the problem and the
    reasons for studying it.~Provide a detailed yet concise background
    discussion of the problem and the significance, scope, and limits of the
    work. Outline what has been done previously by citing truly pertinent
    literature but do not include a general survey of semi-relevant
    literature.~ State how your work differs from earlier work in the field
    and demonstrate the continuity from the previous work to your own.}

    \hypertarget{material-and-methods}{%


        \section{MATERIAL AND METHODS}\label{material-and-methods}}

    \emph{This section is obviously discipline specific so use the
    nomenclature that is common for your discipline. However, this section
    should provide sufficient detail about the materials and the methods
    used so that other experienced workers can repeat the experiment and
    obtain comparable results. Cite the appropriate literature when using a
    standard method or protocol and give only the details needed. Identify
    the materials used in the research. For example, computer systems used,
        mathematical theorems exploited, etc.; give information on the purity of
        all chemicals and reagents employed in the research; include the
        chemical/biological names of all compounds and chemical formulas of
        substances that are new or uncommon. Use standard systematic
        nomenclature to unambiguously define well-established compounds,
        processes, equipment, etc.}

    \hypertarget{results}{%


        \section{RESULTS}\label{results}}

    \emph{Summarize the data collected in this section, and their
    statistical treatment. Include only relevant data, but give sufficient
    detail to justify the conclusions.~ It is appropriate in this section to
    use equations, figures, and tables to display your data. Extensive, but
    relevant data, should be reserved for an appendix where it is identified
    as supporting information.}

    \emph{The table or figure must follow as closely as possible after the
    paragraph in which it is referenced. Titles/captions should be kept
    brief.}

    \protect\hypertarget{_Toc93020061}{}{}Table 1 Parameters for the
    optimization of the principal component analysis for olive oil
    adulteration

    \begin{longtable}[]{@{}llll@{}}
        \toprule
        \emph{Replace} & \emph{With} & \emph{Your} & \emph{Table} \\
        \midrule
        \endhead
        &             &             &              \\
        &             &             &              \\
        \bottomrule
    \end{longtable}

    \includegraphics[width=1.95833in,height=1.46875in]{vertopal_23eb081fc9d74a2dbd905a0fc9194429/media/image1.jpeg}\emph{(Replace
    with your figure)}

    \protect\hypertarget{_Toc93020444}{}{}Figure 1 The notorious BTC
    (Brandon the Cat)

    \hypertarget{discussion}{%


        \section{DISCUSSION}\label{discussion}}

    \emph{The discussion section is where you interpret and compare the
    results. The objective is to point out the features and limitations of
    the work. Relate your results to current knowledge in the field and to
    the original purpose for undertaking the project.}

    \hypertarget{conclusions}{%


        \section{CONCLUSIONS}\label{conclusions}}

    \emph{This section is written to put the interpretation of the results
    into the context of the original problem.~ Do not repeat the discussion
    points or include irrelevant material. The conclusion should be based on
    the evidence presented.}

    \hypertarget{references}{%


        \section{REFERENCES}\label{references}}

    \emph{Many bibliographic styles are acceptable for publications in the
    natural sciences. Only for the sake of having one standard across all
    disciplines, you should use this one:}

    \emph{\underline{Body: Superscripted Number}. \{e.g. Nucleotide excision
    repair (NER) is a versatile, error-free mechanism to identify and remove
    a wide assortment of chemically unrelated lesions.\textsuperscript{18}
    NER can be classified into two sub-pathways based on the way DNA lesions
    are identified.\textsuperscript{19} In transcription coupled NER
        (TC-NER), damaged DNA is identified by the stalling of RNA polymerases
        when they encounter bulky covalent DNA lesions.\textsuperscript{20}}

    \emph{\underline{Journal Article}: Evans, D. A., Fitch, D. M., Smith, T.
    E., Cee, V. J. Application of Complex Aldol Reactions to the Total
    Synthesis of Phorboxazole B.~J. Am. Chem.
    Soc.~\textbf{2000,}~122,~10033-10046.}

    \emph{\underline{Book}: Anastas, P. T., Warner, J. C.~Green Chemistry:
    Theory and Practice; Oxford University Press: Oxford, 1998.}

    \hypertarget{appendices}{%


        \section{APPENDICES}\label{appendices}}

    \protect\hypertarget{_Toc93021181}{}{}\textbf{APPENDIX A: APPENDIX
    TITLE}

\end{document}
